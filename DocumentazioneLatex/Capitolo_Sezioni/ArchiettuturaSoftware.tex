\begin{figure}[h]
	\centering
	\includegraphics[width=0.8\textwidth]{Immagini/Architettura_Software}
	\caption{Architettura Software}
	\label{fig:ArchitetturaSoftware}
\end{figure}
\newpage
Nella figura ~\ref{fig:ArchitetturaSoftware} è mostrata l'architettura software modellizzata attraverso una rete di Petri.

Innanzitutto si può già osservare come essa sia stata definita seguendo il modello architetturale MVC:
\begin{itemize}
	\item A monte dell'intera applicazione è prevista una parte riservata all'interfaccia grafica, attraverso la quale sarà possibile inviare e ricevere informazioni dal server applicativo. Si vede, infatti, che è stata predisposta una comunicazione bidirezionale tra dispositivo Android e Server AWS.
	\item Al centro sono rappresentate tutte le richieste a cui il lato server è in grado di rispondere, ovvero funzioni implementate lato Server.
	\item Infine è prevista una banca dati persistente, in questo caso un database relazionale, al quale il Server Applicativo accede sia per operazioni di lettura che di scrittura, sempre con lo scopo di far fronte alle richieste provenienti dal lato utente.
\end{itemize}
Si può quindi constatare che non si trattano di strati tra loro indipendenti, poichè il flusso dei dati li coinvolge tutti.