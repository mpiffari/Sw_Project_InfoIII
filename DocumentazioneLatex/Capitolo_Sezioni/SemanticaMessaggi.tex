I messaggi inviati dai client, ovvero direttamente dagli users, non sono altro che stringhe le quali possono essere identificate dalla seguente \textit{RegEx} (\textit{Regular Expression}): 

\begin{lstlisting}[caption={\textit{RegEx} utilizzata},captionpos=b]
	^[a-zA-Z0-9]+;[a-zA-Z0-9]+:[0-9]+;([a-zA-Z0-9]+:[a-zA-Z0-9]+;)+$
\end{lstlisting}

la cui semantica può essere così rappresentata:

\begin{center}
"<username>;requestType:<tipo>;<richiesta>"
\end{center}

I valori assunti da <tipo> sono i seguenti e corrispondo ai campi definiti all'interno dell'enumerativo \textit{RequestType}.

Durante la prima iterazione siamo andati a gestire le richieste relative alle sole due seguenti tipologie di richieste:
\begin{itemize}
	\item 0 -> BOOK\_REGISTRATION\_MANUAL
	\item 8 -> BOOK\_SEARCH
\end{itemize}

Nella seconda iterazione, e nelle successive, è stata pianificata l'implementazione della parte di gestione delle restanti richieste:

\begin{itemize}
	\item 1 -> BOOK\_RESERVATION	
	\item 2 -> LOGIN
	\item 3 -> SIGN\_IN
	\item 4 -> BOOK\_REGISTRATION\_AUTOMATIC
	\item 5 -> PROFILE\_INFO
	\item 6 -> TAKEN\_BOOKS
	\item 7 -> PICK\_UP	
\end{itemize}
La richiesta inviata dal client Android verso il server, ad esempio, per la ricerca di un libro assume la seguente forma:

\begin{lstlisting}[caption={Semantica della richiesta per ricerca},captionpos=b]
	username + ";" + "requestType:" + 0 + ";" + book.encode();
\end{lstlisting}

\noindent Lato Android la richiesta vera e propria è preceduta dall'username dell'utente collegato in modo che, lato server, qualora per diversi motivi l'username fosse invalido, la richiesta venga ignorata a priori.
\\ \noindent
Analogamente possiamo identificare i messaggi inviati dal server come:
\begin{lstlisting}[caption={\textit{RegEx} utilizzata lato server per l'interpretazione delle richieste},captionpos=b]
	^[a-zA-Z0-9]+:[0-9]+;([a-zA-Z0-9]+:[a-zA-Z0-9]+;)+$
\end{lstlisting}
la cui semantica può a sua volta essere così rappresentata:
\begin{center}
	"requestType:<tipo>;<risultato>"
\end{center}

\noindent Ad esempio, nel caso di invio di una risposta da parte del server verso il client Android, in merito alla registrazione manuale di un libro andata a buon fine, la stringa va ad assumere il seguente formato:
\begin{lstlisting}[caption={Risposta OK},captionpos=b]
	"requestType:0;result:" + 1 + ";BCID:" + bcid
\end{lstlisting}

dove \textit{bcid} è il codice alfanumerico generato casualmente dal server per identificare il libro all'interno della rete di Book Crossing.

Qual'ora invece il server riceva una richiesta con un tipo errato, la risposta che andrebbe ad inoltrare sarebbe la seguente:
\begin{lstlisting}[caption={Risposta KO},captionpos=b]
	"requestType:10000;result:KO_RequestType"
\end{lstlisting}





