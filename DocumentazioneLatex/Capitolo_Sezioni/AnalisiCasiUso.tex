

\begin{itemize}
	\item \textbf{\textit{UC1: Registrazione}}
	\begin{itemize}
		\item \textbf{Descrizione:} registrazione alla rete di Book Crossing.
		\item \textbf{Attori coinvolti:} utente finale.
		\item \textbf{Preconditions:} 
		\begin{itemize}
			\item smartphone dotato di connessione dati;
			\item l’utente non è ancora registrato al programma di Book Crossing.
		\end{itemize}
		\item \textbf{Postconditions:} l’utente è inserito tra gli utenti registrati al programma di Book Crossing.
		\item \textbf{Processo:}
		\begin{enumerate}
			\item l’utente seleziona “Registrati” nella schermata principale dell’applicazione;
			\item l’applicazione mostra un form da compilare;
			\item l’utente inserisce i dati richiesti nel form presentato;
			\item il sistema verifica la correttezza dei dati inseriti;
			\item l’utente viene reindirizzato alla pagina principale dell’applicazione.
		\end{enumerate}
		\item \textbf{Alternative}
		\item \textbf{Estensioni}
	\end{itemize}
	\item \textbf{\textit{UC2: Login}}
	\begin{itemize}
		\item \textbf{Descrizione:} accesso alla rete di Book Crossing.
		\item \textbf{Attori coinvolti:} utente finale. 
		\item \textbf{Preconditions:} 
		\begin{itemize}
			\item smartphone dotato di connessione dati;
			\item l’utente è già registrato al programma di Book Crossing.
		\end{itemize}
		\item \textbf{Postconditions:} l’utente accede alla rete di Book Crossing
		\item \textbf{Processo:}
		\begin{enumerate}
			\item l’utente seleziona “Login” nella schermata principale dell’applicazione;
			\item l’applicazione mostra una schermata in cui inserire nome utente e password;
			\item l’utente inserisce i dati richiesti nella view presentata;
			\item il sistema verifica la correttezza dei dati inseriti;
			\item l’utente viene reindirizzato alla pagina principale dell’applicazione.
		\end{enumerate}
		\item \textbf{Alternative}
		\item \textbf{Estensioni}
	\end{itemize}
	\item \textit{\textbf{UC3: Raccolta libro}}
	\begin{itemize}
		\item \textbf{Descrizione:} raccolta di un libro “on the go”\footnote{In questo caso il libro condiviso dalla community viene raccolto dall'utente in una qualsiasi zona (come per esempio la stazione, il parco, sale di attesa etc...).} o in una OCZ.\footnote{\textit{Official Crossing Zone}: zone riconosciute e fisse in cui la community può liberamente scambiarsi i libri.}
		\item \textbf{Attori coinvolti:} utente finale.
		\item \textbf{Preconditions:}
		\begin{itemize}
			\item smartphone dotato di connessione dati;
			\item l’utente ha effettuato l’accesso alla rete di Book Crossing;
			\item il libro è stato siglato con il codice BCID.
		\end{itemize}
		\item \textbf{Postconditions:} Il libro viene associato all’utente.
		\item \textbf{Processo:}
		\begin{enumerate}
			\item l’utente seleziona “Raccogli libro” nel menu principale dell’applicazione;
			\item l’applicazione mostra un form da compilare;
			\item l’utente inserisce il codice BCID riportato nel libro;
			\item il sistema verifica la correttezza del codice BCID inserito;
			\item l’applicazione mostra una scheda riepilogativa relativa al libro appena aggiunto;
			\item  l’utente conferma l’operazione.
		\end{enumerate}
		\item \textbf{Alternative}
		\item \textbf{Estensioni}
	\end{itemize}
	\item \textbf{\textit{UC4: Registrazione libro}}
	\begin{itemize}
		\item \textbf{Descrizione:} registrazione di un libro alla rete di Book Crossing (\textit{journal entry}).
		\item \textbf{Generalizzazione:} aggiunta manuale dei dati del libro (U.C. 8) e Scansione ISBN(U.C. 9).
		\item \textbf{Include:} scrittura BCID (U.C. 10).
		\item \textbf{Attori coinvolti:} utente finale.
		\item \textbf{Preconditions:}
		\begin{itemize}
			\item smartphone dotato di connessione dati;
			\item l’utente ha effettuato l’accesso alla rete di Book Crossing;
			\item il libro non è ancora stato siglato con il codice BCID.
		\end{itemize}
		\item \textbf{Postconditions:} Il libro viene aggiunto alla rete di Book Crossing.		
		\item \textbf{Processo:} 
		\begin{enumerate}
			\item l’utente seleziona “Registra un nuovo libro” nel menu principale dell’applicazione;
			\item l’applicazione propone due possibilità: "Scansiona ISBN" e "Aggiunta dati manuale";
			\item l’utente sceglie la proposta preferita;
			\item l’applicazione mostra la schermata relativa alla scelta fatta.
		\end{enumerate}
		\item \textbf{Alternative}
		\item \textbf{Estensioni}
	\end{itemize}
	\item \textbf{\textit{UC5: Ricerca libro}}
	\begin{itemize}
		\item \textbf{Descrizione:} Ricerca di un libro all’interno della rete di Book Crossing.
		\item \textbf{Attori coinvolti:} Utente finale
		\item \textbf{Preconditions:}
		\begin{itemize}
			\item smartphone dotato di connessione dati;
			\item l’utente ha effettuato l’accesso alla rete di Book Crossing;
			\item il libro è stato aggiunto alla rete di Book Crossing;
		\end{itemize}
		\item \textbf{Postconditions:} L'applicazione mostra il libro ricercato.
		\item \textbf{Processo:}
		\begin{enumerate}
			\item l’utente seleziona “Ricerca libro” nel menu principale dell’applicazione;
			\item l’applicazione mostra un form da completare con il titolo del libro da ricercare;
			\item l’utente inserisce nel form presentato il titolo del libro da cercare e conferma;
			\item il sistema verifica la presenza del libro cercato;
			\item in caso di esito positivo, l’applicazione mostra una scheda riassuntiva del libro;
			\item in caso di esito negativo, l’applicazione mostrerà un messaggio di errore;
		\end{enumerate}
		\item \textbf{Alternative}
		\item \textbf{Estensioni}
	\end{itemize}
	\item \textbf{\textit{UC6: Visualizzazione info}}
	\begin{itemize}
		\item \textbf{Descrizione: Visualizzazione informazioni}
		\item \textbf{Generalizzazione:} visualizzazione informazioni libri chased(UC.13), visualizzazione informazioni libri relased(UC.14), visualizzazione informazioni libri in possesso(UC.15)
		\item \textbf{Attori coinvolti:} Utente finale
		\item \textbf{Preconditions:}
		\begin{itemize}
			\item smartphone dotato di connessione dati;
			\item l’utente ha effettuato l’accesso alla rete di Book Crossing;
		\end{itemize}
		\item \textbf{Postconditions:} L’applicazione mostra le informazioni desiderate
		\item \textbf{Processo:}
		\begin{enumerate}
			\item l’utente seleziona la voce “I miei libri” nel menu principale dell’applicazione;
			\item l’applicazione mostra categorie di informazioni visualizzabili;
			\item l’utente seleziona la categoria che vuole visualizzare;
		\end{enumerate}
		\item \textbf{Alternative}
		\item \textbf{Estensioni}
	\end{itemize}
	\item \textbf{\textit{UC7: Visualizzazione profilo personale}}
	\begin{itemize}
		\item \textbf{Descrizione: } Visualizzazione profilo personale utente
		\item \textbf{Attori coinvolti:} Utente finale
		\item \textbf{Preconditions:}
		\begin{itemize}
			\item smartphone dotato di connessione dati;
			\item l’utente ha effettuato l’accesso alla rete di Book Crossing;
		\end{itemize}
		\item \textbf{Postconditions: }L’applicazione mostra il profilo dell’utente.
		\item \textbf{Processo: }
		\begin{enumerate}
			\item l’utente seleziona la voce “Il mio profilo” nel menu principale dell’applicazione;
			\item l’applicazione mostra l’anagrafica, i contatti e le attività svolte dall’utente;
		\end{enumerate}
		\item \textbf{Alternative}
		\item \textbf{Estensioni: } Visualizzazione informazioni libri in possesso(U.C.15)
	\end{itemize}
	\item \textbf{\textit{UC8: Aggiunta manuale dei dati del libro}}
	\begin{itemize}
		\item \textbf{Descrizione:} Inserimento manuale di un libro nella rete di Book Crossing
		\item \textbf{Attori coinvolti:} Utente finale
		\item \textbf{Preconditions:}
		\begin{itemize}
			\item smartphone dotato di connessione dati;
			\item l’utente ha effettuato l’accesso alla rete di Book Crossing;
			\item il libro non è stato ancora siglato con il codice BCID;
		\end{itemize}
		\item \textbf{Postcondition:} Viene generato il codice BCID e il libro viene aggiunto alla rete di Book Crossing
		\item \textbf{Processo: }
		\begin{enumerate}
			\item facendo riferimento al passo 1 e 2 del U.C. 4, scegliere la voce “Aggiunta manuale”;
			\item l’applicazione mostra un form da compilare con i dati del libro;
			\item l’utente inserisce i dati del libro richiesti e conferma l’operazione;
			\item il sistema verifica la correttezza dei dati;
			\item l’applicazione mostra il codice BCID da trascrivere sul libro;
			\item il sistema aggiunge il libro alla rete di Book Crossing;
		\end{enumerate}
		\item \textbf{Alternative}
		\item \textbf{Estensioni}
	\end{itemize}
	\item \textbf{\textit{UC9: Scansione ISBN}}
	\begin{itemize}
		\item \textbf{Descrizione:} scansione del codice ISBN del libro tramite fotocamera per ottenere le informazioni in merito al libro da registrare.
		\item \textbf{Attori coinvolti:} utente finale.
		\item {Preconditions:} 
		\begin{itemize}
			\item smartphone dotato di connessione dati;
			\item l’utente ha effettuato l’accesso alla rete di Book Crossing;
			\item il libro possiede il codice ISBN
		\end{itemize}
		\item \textbf{Postconditions:} il libro è in possesso dell'utente e non più condiviso con la community.
		\item \textbf{Processo:}
		\begin{enumerate}
			\item l’utente seleziona “Registra un nuovo libro” nel menu principale dell’applicazione;
			\item l’applicazione propone due possibilità: "Scansiona ISBN" e "Aggiunta dati manuale";
			\item l'utilizzatore preme il pulsante di "Scansione ISBN";
			\item viene aperta la fotocamera all'interno dell'applicazione;
			\item l'utente inquadra il codice ISBN finchè il sistema non chiude la automaticamente la fotocamera, rielaborando i dati acquisiti.
		\end{enumerate}
		\item \textbf{Alternative}
		\begin{itemize}
			\item \textbf{ISBN non riconsciuto:} il sistema non è in grado di riconoscere l'ISBN inquadrato. Si chiuderà la fotocamera e l'utente verrà reindirizzato avvisato di questo malfunzioanmento tramite UI.
		\end{itemize}
		\item \textbf{Estensioni}
	\end{itemize}
	\item \textbf{\textit{UC10: Scrittura BCID\footnote{\textit{Book Crossing IDentifier}}}}
	\begin{itemize}
		\item \textbf{Descrizione:} scrittura del codice identificativo sul libro condiviso.
		\item \textbf{Attori coinvolti:} utente finale.
		\item \textbf{Preconditions:}
		\begin{itemize}
			\item smartphone dotato di connessione dati;
			\item l’utente ha effettuato l’accesso alla rete di Book Crossing;
			\item l'utente deve essere in possesso di un libro.
		\end{itemize}
		\item \textbf{Postconditions:} il libro è univocamente riconosciuto del sistema tramite il BCID.
		\item \textbf{Processo:} l'utente copia il codice BCID dopo aver registrato il libro al sistema.
		\item \textbf{Alternative}
		\item \textbf{Estensioni}
	\end{itemize}
	\item \textbf{\textit{UC11: Visualizzazione contatti utente}}
	\begin{itemize}
		\item \textbf{Descrizione:} l'utente ottiene i contatti che un certo altro utilizzatore ha deciso di condividere con il sistema.
		\item \textbf{Attori coinvolti:} utente finale.
		\item \textbf{Preconditions:}
		\begin{itemize}
			\item smartphone dotato di connessione dati;
			\item l’utente ha effettuato l’accesso alla rete di Book Crossing.
		\end{itemize}
		\item \textbf{Postconditions:} l'applicazione visualizza i contatti a cui l'utente terzo può/vuole essere contattato.
		\item \textbf{Processo:}
		\begin{enumerate}
			\item l'utente ricerca un libro dal menu principale dell'applicazione;
			\item l'utente seleziona il libro;
			\item il sistema mostra tutte le informazioni relative al libro (tra cui anche la lista di tutti i lettori che sono stati in possesso del libro in questione);
			\item l'utlizzatore seleziona l'utente;
			\item il sistema mostra tutte le informazioni di contatto che il cliente terzo ha deciso di condividere con la community.
		\end{enumerate}
		\item \textbf{Alternative:}
		\begin{itemize}
			\item \textbf{Nessun libro trovato:} la grafica notifica l'utilizzatore del fatto che la ricerca non sia andata a buon fine.
			\item \textbf{Utente senza alcun contatto condiviso:} il sistema filtra la visualizzazione della lista degli utenti, visualizzando solo quegli utenti che hanno inserito, durante la fase di registrazione, almeno un contatto.
		\end{itemize}
		\item \textbf{Estensioni}
	\end{itemize}
	\item \textbf{\textit{UC12: Prenotazione libro}}
	\begin{itemize}
		\item \textbf{Descrizione:} l'utilizzatore esprime il desiderio di ricevere un determinato libro da parte di un altro lettore. Per cercare di far arrivare il libro in questione all'utente richiedente, esso può aprire una prenotazione verso l'utente terzo che attualmente è in possesso del volume.
		\item \textbf{Attori coinvolti:} 
		\begin{itemize}
			\item utente richiedente \textit{claimant user};
			\item utente attualmente in possesso del libro richiesto \textit{owner user}.
		\end{itemize}
		\item \textbf{Preconditions:}
		\begin{itemize}
			\item smartphone dotato di connessione dati;
			\item l’utente ha effettuato l’accesso alla rete di Book Crossing;
			\item il libro che si vuole prenotare deve essere registrato alla rete;
			\item il libro richiesto deve essere già in possesso di un altro utente;
		\end{itemize}
		\item \textbf{Postconditions:}
		\begin{itemize}
			\item \textit{Owner user} viene consigliato dal sistema dove rilasciare il libro, nel caso in cui lo voglia fare, in maneira da avvicinare il libro al \textit{claimant user}
			\item \textit{claimant user} viene notificato dal sistema nel momento in cui il libro risulta essere nelle sue vicinanze
		\end{itemize}
		\item \textbf{Processo:}
		\begin{enumerate}
			\item l'utente ricerca, dal menù principale dell'applicazione, il titolo del libro oppure il nome dell'autore;
			\item l'applicazione mostrerà una lista di tutti i libri con quel determinato titolo o di quello specifico autore, visualizzando anche la località in cui si trova;
			\item l'utente va a selezionare il libro all'interno della lista proposta dal sistema;
			\item l'applicazione mostra un riepilogo sulle informazioni del libro, unitamente alla possibilità di prenotare;
			\item l'utente preme il pulsante "Prenota";
			\item viene inoltrata la richiesta all' \textit{owner user} unitamente alla visualizzazione dei suoi contatti.
		\end{enumerate}
		\item \textbf{Alternative}
		\item \textbf{Estensioni}
	\end{itemize}
	\item \textbf{\textit{UC13: Visualizzazione info libri chased}}
	\begin{itemize}
		\item \textbf{Descrizione:} visualizzazione storico dei libri "raccolti" dall'utente.
		\item \textbf{Attori coinvolti:} utente finale.
		\item \textbf{Preconditions:}
		\begin{itemize}
			\item smartphone dotato di connessione dati;
			\item l’utente ha effettuato l’accesso alla rete di Book Crossing.
		\end{itemize}
		\item \textbf{Postconditions:} mostrata sulla grafica la lista dei libri raccolti, con relativa data e luogo di "chasing".
		\item \textbf{Processo:}
		\begin{enumerate}
			\item l'utente, dal menù principale dell'applicazione, entra nella propria area riservata;
			\item l'applicazione mostrerà un riassunto delle informazioni dell'utilizzatore;
			\item l'utente va preme il pulsante "Libri chased";
			\item l'applicazione mostra un riepilogo di tutti i testi ottenuti dalla community di sharing.
		\end{enumerate}
		\item \textbf{Alternative:}
		\begin{itemize}
			\item \textbf{Nessun libro chased:} la grafica notifica l'utilizzatore del fatto che non sia ancora stata effettuata una raccolta di almeno un volume.
		\end{itemize}
		\item \textbf{Estensioni}
	\end{itemize}
	\item \textbf{\textit{UC14: Visualizzazione info libri released}}
	\begin{itemize}
		\item \textbf{Descrizione:} visualizzazione storico dei propri libri condivisi con la rete.
		\item \textbf{Attori coinvolti:}  utente finale.
		\item \textbf{Preconditions:}
		\begin{itemize}
			\item smartphone dotato di connessione dati;
			\item l’utente ha effettuato l’accesso alla rete di Book Crossing.
		\end{itemize}
		\item \textbf{Postconditions:} mostrata sulla grafica la lista dei libri inseriti nel programma di sharing, con relativa data e luogo di "relase".
		\item \textbf{Processo:}
		\begin{enumerate}
			\item l'utente, dal menù principale dell'applicazione, entra nella propria area riservata;
			\item l'applicazione mostrerà un riassunto delle informazioni dell'utilizzatore;
			\item l'utente va preme il pulsante "Libri released";
			\item l'applicazione mostra un riepilogo di tutti i libri rilasciati alla community di sharing.
		\end{enumerate}
		\item \textbf{Alternative:}
		\begin{itemize}
			\item \textbf{Nessun libro released:} la grafica notifica l'utilizzatore del fatto che non sia ancora stata effettuato un rilascio di almeno un volume.
		\end{itemize}
		\item \textbf{Estensioni}
	\end{itemize}
	\item \textbf{\textit{UC15: Visualizzazione info libri in possesso}}
	\begin{itemize}
		\item \textbf{Descrizione:} visualizzazione dei propri libri attualmente "under reading".
		\item \textbf{Attori coinvolti:}  utente finale.
		\item \textbf{Preconditions:}
		\begin{itemize}
			\item smartphone dotato di connessione dati;
			\item l’utente ha effettuato l’accesso alla rete di Book Crossing.
		\end{itemize}
		\item \textbf{Postconditions:} mostrata sulla grafica la lista dei libri attualmente in possesso dell'utente stesso. 
		\item \textbf{Processo:}
		\begin{enumerate}
			\item l'utente, dal menù principale dell'applicazione, entra nella propria area riservata;
			\item l'applicazione mostrerà un riassunto delle informazioni dell'utilizzatore;
			\item l'utente va preme il pulsante "Libri in possesso";
			\item l'applicazione mostra un riepilogo di tutti i libri attualmente in possesso.
		\end{enumerate}
		\item \textbf{Alternative:}
		\begin{itemize}
			\item \textbf{Nessun libro in lettura:} la grafica notifica l'utilizzatore del fatto che non ha alcun libro in proprio possesso.
		\end{itemize}
		\item \textbf{Estensioni}
	\end{itemize}
	\item \textbf{\textit{UC16: Rilascio libro}}
	\begin{itemize}
		\item \textbf{Descrizione:} l'utente libera un libro.
		\item \textbf{Attori coinvolti:} utente finale.
		\item \textbf{Preconditions:}
		\begin{itemize}
			\item smartphone dotato di connessione dati;
			\item l’utente ha effettuato l’accesso alla rete di Book Crossing;
			\item libro rilasciato già registrato al sistema di Book Crossing.
		\end{itemize}
		\item \textbf{Postconditions:} il libro passa dallo stato "under reading" a quello "free".
		\item \textbf{Processo:}
		\begin{enumerate}
			\item l'utente, dal menù principale dell'applicazione, entra nella propria area riservata;
			\item l'applicazione mostrerà un riassunto delle informazioni dell'utilizzatore;
			\item l'utente va preme il pulsante "Libri in possesso";
			\item l'applicazione mostra un riepilogo di tutti i libri attualmente in possesso;
			\item  l'utente preme sul testo che intende rilasciare;
			\item il sistema suggerisce all'utente dove rilasciare il testo, per facilitare la creazione di cassette virtuali;
			\item l'utente conferma il rilascio.
		\end{enumerate}
		\item \textbf{Alternative:}
		\begin{itemize}
			\item \textbf{Nessun libro in lettura:} la grafica notifica l'utilizzatore del fatto che non ha alcun libro in proprio possesso.
			\item \textbf{Segnale GPS non trovato:} il sistema avvisa l'utente di andare ad attivare il segnale GPS.
		\end{itemize}
		\item \textbf{Estensioni}
	\end{itemize}
\end{itemize}