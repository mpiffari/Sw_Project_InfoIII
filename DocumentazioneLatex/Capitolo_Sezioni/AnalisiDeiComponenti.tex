Come già presentato in precedenza, per la definizione dei componenti si è deciso di seguire il pattern architetturale MVC. 
Le componenti che si è deciso di sviluppare durante la prima iterazione sono: 
\begin{itemize}
	\item \textbf{Componente \textit{Manual Book registration:}} la componente \textit{Manual Book registration} fa riferimento all’ UC9 (~\ref{itemize:UC9} ), figlio del caso d'uso più generico UC5 (~\ref{itemize:UC5} ), ovvero alla funzione di aggiunta di un libro alla rete di Book Crossing per via manuale. 
	
	La componente si presenta nel seguente modo:
	\begin{itemize}
		\item \textit{GUI:} interfaccia grafica utilizzata per registrare un libro alla rete di Book Crossing. Verranno quindi messe a disposizione una serie di interfaccie grafiche, composte sostanzialmente da campi da compilare, per aggiungere le informazioni relative al proprio libro, ottenendo poi, successivamente alla registrazione, il relativo BCID;
		\item \textit{Model:} si fa carico di ricevere le informazioni relative al libro e, sfruttando la parte \textit{Data}, restituisce alla parte \textit{GUI} il BCID con il quale siglare il libro;
		\item \textit{Data:} le informazioni relative al libro che si vuole aggiungere sono memorizzate nel Database RDS, associandolo all'utente che attualmente lo possiede. 
	\end{itemize}
	\item \textbf{Componente \textit{Ricerca:}}  la componente di \textit{Ricerca} fa riferimento all’ UC6 (~\ref{itemize:UC6} ),
	ovvero alla funzione che permette di andare a ricercare un libro all'interno della piattaforma di Book crossing.
	Questa ricerca può avvenire per titolo, per autore oppure sia per autore che per titolo. Il componente si presenta nel seguente modo:
	\begin{itemize}
		\item \textit{GUI:} interfaccia grafica composta da due textbox in cui andare ad inserire titolo e/o autore. Essendo possibili tre tipologie di ricerca, come specificato in precedenza, non è necessario compilare entrambi i campi (lo è solamente nel caso in cui si è interessati a compiere una ricerca basandosi su entrambi i vincoli);
		\item \textit{Model:} ad esso compete la parte relativa allo smistamento delle richieste, a seconda del fatto che si stia eseguendo una ricerca per titolo, autore o per entrambi;
		\item \textit{Data:} fornisce, se presenti, le informazioni relative al libro oggetto della ricerca.
	\end{itemize}
\end{itemize}