Per la realizzazione del software presentato in questo report sono stati utilizzati i seguenti tool:
\begin{itemize}
	\item \textbf{Modellazione}
	\begin{itemize}
		\item \textbf{Use case diagram}: realizzato con il tool online \textit{Lucid Chart} (\href{https://www.lucidchart.com}{https://www.lucidchart.com});
		\item \textbf{Database architecture}: realizzato con \textit{Vertabelo} (\href{https://www.vertabelo.com/}{https://www.vertabelo.com/});
		\item \textbf{Class diagram - deployment diagram - logical view}: EDrawMax (\href{https://www.edrawsoft.com/en/edraw-max/}{https://www.edrawsoft.com/en/edraw-max/}) e Visual Paradigm (\href{https://www.visual-paradigm.com/}{https://www.visual-paradigm.com/})
	\end{itemize}
	\item \textbf{Implementazione software}
	\begin{itemize}
		\item \textbf{Eclipse}: per quanto riguarda l'implementazione del codice lato server.
		\item \textbf{Android Studio}: ambiente di sviluppo utilizzato per lo sviluppo della parte mobile.
	\end{itemize}
	
	\item \textbf{Analisi del software}
	\begin{itemize}
		\item \textbf{Espresso - JUnit} per l'analisi dinamica del codice
		\item \textbf{CodeCover - SpotBugs} per l'analisi statica
	\end{itemize}
	\item \textbf{Tool vari}
	\begin{itemize}
		\item \textbf{Versioning}: repository Github gestito tramite interfaccia grafica;
		\item \textbf{Documentazione}: \LaTeX  tramite interfaccia grafica TeXstudio.
	\end{itemize}
\end{itemize}