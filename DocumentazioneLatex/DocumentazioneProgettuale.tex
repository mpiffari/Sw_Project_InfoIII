%%%%%%%%%%%%%%%%%%%%%%%%%%%%%%%%%%%%%%%%%%%%%%%%%%%
%% LaTeX book template                           %%
%% Author:  Amber Jain (http://amberj.devio.us/) %%
%% License: ISC license                          %%
%%%%%%%%%%%%%%%%%%%%%%%%%%%%%%%%%%%%%%%%%%%%%%%%%%%

\documentclass[a4paper,11pt]{book}
\usepackage[T1]{fontenc}
\usepackage[utf8]{inputenc}
\usepackage{lmodern}
%%%%%%%%%%%%%%%%%%%%%%%%%%%%%%%%%%%%%%%%%%%%%%%%%%%%%%%%%
% Source: http://en.wikibooks.org/wiki/LaTeX/Hyperlinks %
%%%%%%%%%%%%%%%%%%%%%%%%%%%%%%%%%%%%%%%%%%%%%%%%%%%%%%%%%
\usepackage{hyperref}
\usepackage{graphicx}
\usepackage[english]{babel}
\usepackage[a4paper,top=2cm,bottom=2cm,left=2cm,right=2cm]{geometry}
\usepackage{lscape}
\usepackage{caption}
\usepackage{amsmath}
\usepackage{listingsutf8}
\usepackage[ruled]{algorithm2e}

\lstset{% general command to set parameter(s)
	basicstyle=\small, % print whole listing small
	numbers=left,
	keywordstyle=\color{black}\bfseries,
	% underlined bold black keywords
	identifierstyle=, % nothing happens
	stringstyle=\ttfamily} % typewriter type for strings

\lstset{language=Java} 
\captionsetup{tableposition=top,figureposition=bottom,font=small}
%%%%%%%%%%%%%%%%%%%%%%%%%%%%%%%%%%%%%%%%%%%%%%%%%%%%%%%%%%%%%%%%%%%%%%%%%%%%%%%%
% 'dedication' environment: To add a dedication paragraph at the start of book %
% Source: http://www.tug.org/pipermail/texhax/2010-June/015184.html            %
%%%%%%%%%%%%%%%%%%%%%%%%%%%%%%%%%%%%%%%%%%%%%%%%%%%%%%%%%%%%%%%%%%%%%%%%%%%%%%%%
\newenvironment{dedication}
{
   \cleardoublepage
   \thispagestyle{empty}
   \vspace*{\stretch{1}}
   \hfill\begin{minipage}[t]{0.66\textwidth}
   \raggedright
}
{
   \end{minipage}
   \vspace*{\stretch{3}}
   \clearpage
}

%%%%%%%%%%%%%%%%%%%%%%%%%%%%%%%%%%%%%%%%%%%%%%%%
% Chapter quote at the start of chapter        %
% Source: http://tex.stackexchange.com/a/53380 %
%%%%%%%%%%%%%%%%%%%%%%%%%%%%%%%%%%%%%%%%%%%%%%%%
\makeatletter
\renewcommand{\@chapapp}{}% Not necessary...
\newenvironment{chapquote}[2][2em]
  {\setlength{\@tempdima}{#1}%
   \def\chapquote@author{#2}%
   \parshape 1 \@tempdima \dimexpr\textwidth-2\@tempdima\relax%
   \itshape}
  {\par\normalfont\hfill--\ \chapquote@author\hspace*{\@tempdima}\par\bigskip}
\makeatother

%%%%%%%%%%%%%%%%%%%%%%%%%%%%%%%%%%%%%%%%%%%%%%%%%%%
% First page of book which contains 'stuff' like: %
%  - Book title, subtitle                         %
%  - Book author name                             %
%%%%%%%%%%%%%%%%%%%%%%%%%%%%%%%%%%%%%%%%%%%%%%%%%%%

% Book's title and subtitle
\title{\Huge \textbf{Gestione di un sistema di book-crossing:\\ \textit{\textbf{The WalkingBooks}}} \\ \huge \textit{\textbf{Documentazione progettuale}} \\ \bigskip \huge Progetto del corso di Informatica IIIB \\ \huge A.A. 2018/2019}
% Author
\author{\textsc{Paganessi Andrea} \\ \textsc{Piffari Michele} \\ \textsc{Villa Stefano}}


\begin{document}

\frontmatter
\maketitle

%%%%%%%%%%%%%%%%%%%%%%%%%%%%%%%%%%%%%%%%%%%%%%%%%%%%%%%%%%%%%%%%%%%%%%%%
% Auto-generated table of contents, list of figures and list of tables %
%%%%%%%%%%%%%%%%%%%%%%%%%%%%%%%%%%%%%%%%%%%%%%%%%%%%%%%%%%%%%%%%%%%%%%%%
\tableofcontents
\listoffigures
\listoftables

\mainmatter

%%%%%%%%%%%%%%%%
% NEW PART! %
%%%%%%%%%%%%%%%%
\part{Iterazione 0}

%%%%%%%%%%%%%%%%
% NEW CHAPTER! %
%%%%%%%%%%%%%%%%
\chapter{Requisiti e specifiche}
\section{Requisiti utente}
La startUp bergamasca Book Crossing UniBg desidera mettere a disposizione dei propri
utenti un'applicazione Android per poter gestire la libera condivisione di libri all'interno di una vasta community di utenti.

I libri della rete di BookCrossing che l'azienda punta a gestire si possono trovare
\begin{itemize}
	\item Casualmente (in stazione, su una panchina, in un locale o in un qualsiasi altro luogo): funzionalità \textit{"on the go"}
	\item Nella zona di scambio ufficiale (\textit{"OCZ UniBg"}: Official Crossing Zone UniBg)
\end{itemize}

Per il momento, l'unica OCZ gestita direttamente dalla startUp si trova all'interno dell'aula studio del campus di Ingegneria di Dalmine, la quale coincide anche con la sistemazione del server centrale che andrà a gestire i vari interscambi tra gli users.

La startUp richiede che, per usufruire dell'applicativo mobile, i clienti debbano registrarsi fornendo i propri dati quali
\begin{itemize}
	\item Nome
	\item Cognome
	\item Contatto di riferimento (opzionale)
	\begin{itemize}
		\item Numero telefonico
		\item Indirizzo mail
		\item Facebook
		\item ID Twitter
	\end{itemize}
	\item Categorie di libri preferite
	\item Zona di residenza
\end{itemize}

Una volta registratosi, l'utente può partecipare al programma di Book Crossing.

Secondo la politica del book sharing, per rendere disponibile alla comunità 
uno o più libri che non sono ancora presenti nel network stesso, 
serve
identificarli univocamente, per poterne così tracciare la storia, ovvero ciò 
che concerne il percorso seguito dal libro, le recensioni lasciate dagli utenti etc.

Prima di procedere con l'identificazione univoca del libro, l'utilizzatore deve inserire i dati 
del testo (o dei testi) che intende condividere con il resto della community: questo inserimento può avvenire
- In maniera "automatica" tramite scansione del codice ISBN
- In modalità "manuale", nel caso in cui, per esempio, non sia presente il barcode, fornendo i 
seguenti dati:
\begin{itemize}
	\item Titolo
	\item Autore
	\item Anno di pubblicazione/Edizione
	\item Categoria
\end{itemize}

A questo punto il sistema genererà un BCID di 10 caratteri, ovvero un \textit{Book Crossing ID} univoco, il
quale dovrà essere riportato sul testo dall'utente.


La vera e propria condivisione avviene nel momento in cui il volume/i viene rilasciato (azione che può avvenire in un secondo momento
rispetto alla fase di identificazione), il sistema dovrà acquisire i seguenti dati:
\begin{itemize}
	\item Luogo di rilascio (con estensione future per un'acquisizione automatica della posizione tramite GPS)
	\item Ora e data di rilascio 
\end{itemize}

L'app inoltre consiglierà all'utente un luogo di rilascio in cui sia già presente almeno un libro, 
facilitando così la creazione di cassette virtuali, ovvero di luoghi in cui sono presenti più libri: 
l'idea è quella quindi di permettere al sistema di creare, in maniera autonoma, dei punti "fissi" di 
consegna senza dover applicare interventi a livello infrastrutturale.

Successivamente il sistema dovrà notificare gli utenti, interessati al genere del 
libro rilasciato, della presenza di un nuovo testo, appena rilasciato, che potrebbe interessargli.

In qualsiasi momento è possibile effettuare le seguenti operezioni su ogni libro personalmente 
condiviso con la rete di sharing:
\begin{itemize}
	\item Aggiunta di recensione
	\item Rating del libro
\end{itemize}

Quando viene trovato un libro (nel gergo definito come \textit{"journal entry"}), il cliente che vuole prelevarlo, dopo
aver effettuato il login nell'applicazione, deve inserire nell'apposito menù il BCID del libro che intende acquisire.
Il sistema si occuperà poi di informare la community aggiornando lo status del libro raccolto, che diventerà "underReading".

Per quanto concerne invece l'area riservata, ogni utente ha la possibilità di 
visualizzare informazioni in merito ai libri che: 
\begin{itemize}
	\item ha messo a disposizione della community (\textbf{relased})
	\item ha ottenuto dalla community (\textbf{chased})
	\item attualmente possiede
\end{itemize}

L'utente può effettuare la prenotazione di libri già inseriti nella liste "chased" e "relased" del proprio profilo.


Il sistema deve prevedere anche la possibilità di ricercare un specifico testo e visualizzare i contatti dei
lettori del libro al fine di potersi scambiare opinioni e/o pareri in merito al libro stesso.
Tale funzionalità di ricerca permette anche la prenotazione del testo ricercato purché lo stesso sia nello
stato "under reading".
Per soddisfare questa richiesta il sistema provvederà a consigliare, al lettore corrente 
del libro prenotato, zone di rilascio specifiche al fine di avvicinare tale libro al richiedente, tenendo presente anche la necessità di creare cassette virtuali (come specificato in precedenza).
\chapter{Use cases}
\section{Analisi testuale dei casi d'uso}


\begin{itemize}
	\item \textbf{\textit{UC1: Registrazione}}
	\begin{itemize}
		\item \textbf{Descrizione} 
		\item \textbf{Attori coinvolti} 
		\item \textbf{Obbiettivo}
		\item \textbf{Preconditions}
		\item \textbf{Postconditions}
		\item \textbf{Processo}
		\item \textbf{Alternative}
		\item \textbf{Estensioni}
	\end{itemize}
	\item \textbf{\textit{UC2: Login}}
	\begin{itemize}
		\item \textbf{Descrizione}
		\item \textbf{Attori coinvolti}
		\item \textbf{Obbiettivo}
		\item \textbf{Preconditions}
		\item \textbf{Postconditions}
		\item \textbf{Processo}
		\item \textbf{Alternative}
		\item \textbf{Estensioni}
	\end{itemize}
	\item \textit{\textbf{UC3: Raccolta libro}}
	\begin{itemize}
		\item \textbf{Descrizione}
		\item \textbf{Attori coinvolti}
		\item \textbf{Obbiettivo}
		\item \textbf{Preconditions}
		\item \textbf{Postconditions}
		\item \textbf{Processo}
		\item \textbf{Alternative}
		\item \textbf{Estensioni}
	\end{itemize}
	\item \textbf{\textit{UC4: Registrazione libro}}
	\begin{itemize}
		\item \textbf{Descrizione}
		\item \textbf{Attori coinvolti}
		\item \textbf{Obbiettivo}
		\item \textbf{Preconditions}
		\item \textbf{Postconditions}
		\item \textbf{Processo}
		\item \textbf{Alternative}
		\item \textbf{Estensioni}
	\end{itemize}
	\item \textbf{\textit{UC5: Ricerca libro}}
	\begin{itemize}
		\item \textbf{Descrizione}
		\item \textbf{Attori coinvolti}
		\item \textbf{Obbiettivo}
		\item \textbf{Preconditions}
		\item \textbf{Postconditions}
		\item \textbf{Processo}
		\item \textbf{Alternative}
		\item \textbf{Estensioni}
	\end{itemize}
	\item \textbf{\textit{UC6: Visualizzazione info}}
	\begin{itemize}
		\item \textbf{Descrizione}
		\item \textbf{Attori coinvolti}
		\item \textbf{Obbiettivo}
		\item \textbf{Preconditions}
		\item \textbf{Postconditions}
		\item \textbf{Processo}
		\item \textbf{Alternative}
		\item \textbf{Estensioni}
	\end{itemize}
	\item \textbf{\textit{UC7: Visualizzazione profilo personale}}
	\begin{itemize}
		\item \textbf{Descrizione}
		\item \textbf{Attori coinvolti}
		\item \textbf{Obbiettivo}
		\item \textbf{Preconditions}
		\item \textbf{Postconditions}
		\item \textbf{Processo}
		\item \textbf{Alternative}
		\item \textbf{Estensioni}
	\end{itemize}
	\item \textbf{\textit{UC8: Aggiunta manuale dei dati del libro}}
	\begin{itemize}
		\item \textbf{Descrizione}
		\item \textbf{Attori coinvolti}
		\item \textbf{Obbiettivo}
		\item \textbf{Preconditions}
		\item \textbf{Postconditions}
		\item \textbf{Processo}
		\item \textbf{Alternative}
		\item \textbf{Estensioni}
	\end{itemize}
	\item \textbf{\textit{UC9: Scansione ISBN}}
	\begin{itemize}
		\item \textbf{Descrizione:} Scansione del codice ISBN del libro tramite fotocamera.
		\item \textbf{Attori coinvolti:} Utente finale
		\item \textbf{Obbiettivo:}
		\item \textbf{Preconditions:} l'utente deve essere registrato
		\item \textbf{Postconditions:} il libro è condiviso con la community
		\item \textbf{Processo:} Di seguito andiamo a descrivere il processo
		\begin{enumerate}
			\item 
		\end{enumerate}
		\item \textbf{Alternative}
		\item \textbf{Estensioni}
	\end{itemize}
	\item \textbf{\textit{UC10: Scrittura BCID}}
	\begin{itemize}
		\item \textbf{Descrizione}
		\item \textbf{Attori coinvolti}
		\item \textbf{Obbiettivo}
		\item \textbf{Preconditions}
		\item \textbf{Postconditions}
		\item \textbf{Processo}
		\item \textbf{Alternative}
		\item \textbf{Estensioni}
	\end{itemize}
	\item \textbf{\textit{UC11: Visualizzazione contatti utente}}
	\begin{itemize}
		\item \textbf{Descrizione}
		\item \textbf{Attori coinvolti}
		\item \textbf{Obbiettivo}
		\item \textbf{Preconditions}
		\item \textbf{Postconditions}
		\item \textbf{Processo}
		\item \textbf{Alternative}
		\item \textbf{Estensioni}
	\end{itemize}
	\item \textbf{\textit{UC12: Prenotrazione libro}}
	\begin{itemize}
		\item \textbf{Descrizione}
		\item \textbf{Attori coinvolti}
		\item \textbf{Obbiettivo}
		\item \textbf{Preconditions}
		\item \textbf{Postconditions}
		\item \textbf{Processo}
		\item \textbf{Alternative}
		\item \textbf{Estensioni}
	\end{itemize}
	\item \textbf{\textit{UC13: Visualizzazione info libri chased}}
	\begin{itemize}
		\item \textbf{Descrizione}
		\item \textbf{Attori coinvolti}
		\item \textbf{Obbiettivo}
		\item \textbf{Preconditions}
		\item \textbf{Postconditions}
		\item \textbf{Processo}
		\item \textbf{Alternative}
		\item \textbf{Estensioni}
	\end{itemize}
	\item \textbf{\textit{UC14: Visualizzazione info libri released}}
	\begin{itemize}
		\item \textbf{Descrizione}
		\item \textbf{Attori coinvolti}
		\item \textbf{Obbiettivo}
		\item \textbf{Preconditions}
		\item \textbf{Postconditions}
		\item \textbf{Processo}
		\item \textbf{Alternative}
		\item \textbf{Estensioni}
	\end{itemize}
	\item \textbf{\textit{UC15: Visualizzazione info libri in possesso}}
	\begin{itemize}
		\item \textbf{Descrizione}
		\item \textbf{Attori coinvolti}
		\item \textbf{Obbiettivo}
		\item \textbf{Preconditions}
		\item \textbf{Postconditions}
		\item \textbf{Processo}
		\item \textbf{Alternative}
		\item \textbf{Estensioni}
	\end{itemize}
	\item \textbf{\textit{UC16: Rilascio libro}}
	\begin{itemize}
		\item \textbf{Descrizione}
		\item \textbf{Attori coinvolti}
		\item \textbf{Obbiettivo}
		\item \textbf{Preconditions}
		\item \textbf{Postconditions}
		\item \textbf{Processo}
		\item \textbf{Alternative}
		\item \textbf{Estensioni}
	\end{itemize}
\end{itemize}
\section{Use Case Diagram}
\begin{figure}[h]
	\includegraphics[width=\textwidth]{Immagini/UseCase_BookCrossing}
	\caption{Use cases diagram}
	\label{fig:UsecasesDiagram}
\end{figure}
\newpage
\section{Funzionalità richieste}
\begin{table}
\caption{Panoramica requisiti funzionali progettuali}
\label{tab:Req_utente}
\centering
\begin{tabular}{|l|l|l|l|l|l|r}
	\hline\hline
	\textbf{Nome requisito} & \textbf{ID requisito} & \textbf{Tipologia} & \textbf{Priorità} & \textbf{Requisiti padre} & \textbf{Requisiti figli} \\
	\hline\hline
	Raccolta libro & UR1 & funzionale & alta & & UR2\\
	\hline
	Login utente & UR2 & funzionale & alta & UR1 & UR3, UR4\\
	\hline
	Registrazione utente & UR3 & funzionale & alta & UR2 & \\
	\hline
	Aggiunta libro & UR4 & funzionale & alta & UR2 & UR7, UR9 \\
	\hline
	Ricerca libro & UR5 & funzionale & media & UR2 & UR10 \\
	\hline
	Prenotazione libro & UR6 & funzionale & bassa & UR2 &\\
	\hline
	Visualizzazione info libri chased & UR7 & funzionale & bassa & UR2 &\\
	\hline
	Visualizzazione info libri released & UR8 & funzionale & bassa & UR2, UR4 &\\
	\hline
	Rilascio libro & UR9 & funzionale & alta & UR2, UR4 &\\
	\hline
	Visualizzazione contatti utenti & UR10 & funzionale & bassa & UR2, UR5 &\\
	\hline
	Visualizzazione profilo personale  & UR11 & funzionale & media & UR2 &\\
	\hline
\end{tabular}
\end{table}

\begin{table}
	
	\caption{Descrizione requisiti funzionali progettuali}
	\label{tab:Req_utente_descrizione}
	\centering
	\begin{tabular}{|l|l|l}
		\hline\hline
		\textbf{Nome requisito} & \textbf{ID requisito} & \textbf{Descrizione} \\
		\hline\hline
		Controllo geolocalizzazione & UR12 & Requisito non funzionalità che permette di verificare che la posizione GPS
		salvata del libro corrisponda, con margine d'accettazione, alla posizione 
		in cui si trova l'utente nel momento in cui vuole raccogliere
		un libro trovato "on the go".\\
		\hline
	\end{tabular}
\end{table}
\section{Stati del libro}
La starup si impone l'obbiettivo di andare a gestire lo scambio di libri all'interno della rete di book-crossing: come visto all'interno dei diversi use-cases, ogni libro, nel corso della propria vita all'interno della community, passa di mano in mano, attraversando diverse zone.
A questo movimento fisico, corrisponde anche un continuo cambio di stato da parte del libro stesse: possiamo riassumere con una \textit{"Finite State Machine"} il percorso che un generico libro segue durante la sua vita.

{\LARGE INSERIRE PICCOLO SCHEMA DEGLI STATI}


Riassumendo gli stati di un libro, possono essere:
\begin{itemize}
	\item Out of the network
	\item Available
	\item Under reading
	\item Released
	\item Reserved
	\item Traveling
\end{itemize}

\chapter{Architettura}
\section{Deployment diagram}
\begin{figure}[h]
	\includegraphics[width=\textwidth]{Immagini/Deployment_Diagram}
	\caption{Deployment Diagram}
	\label{fig:Deployment Diagram}
\end{figure}
\section{Architecture Envisioning}
\begin{figure}[h]
	\includegraphics[width=\textwidth]{Immagini/Architecture_Envisoring}
	\caption{Architecture Envisioning}
	\label{fig:ArchitectureEnvisoring}
\end{figure}
\noindent
In figura ~\ref{fig:Deployment Diagram} e ~\ref{fig:ArchitectureEnvisoring} sono mostrati il Deployment Diagram e l'Architecture Envisioning del sistema progettato per lo sviluppo dell’applicazione di Book Crossing. Si può osservare che si tratta di un’architettura \textbf{\textit{Three Tiers}}:
\begin{enumerate}
	\item A sinistra si individua il client, ovvero il dispositivo Android con il quale è possibile interfacciarsi. Al suo interno quindi si può osservare la presenza di un componente relativo all’interfaccia grafica e uno relativo alla gestione delle richieste per invio e ricezione di dati con il server;
	\item Nella parte centrale individuiamo gli altri due layer dell’architettura: server EC2 e Database Relazionale RDS. il fatto di utilizzare Amazon Web Services consente di avere questi due elementi a bordo di un unico strato.
\end{enumerate}
\noindent
Per quanto riguarda la comunicazione tra i vari layer si sfruttano protocolli e librerie fornite sempre dall'ambiente Amazon Web Services. Nel caso della comunicazione tra smartphone e server EC2 si sfrutta Amazon SDK(nota con link), mentre per interfacciarsi con il Database RDS si sfrutta il connettore JDBC.
\part{Iterazione 1}
\chapter{Architettura}
\section{Architettura software}
\begin{figure}[h]
	\includegraphics[width=\textwidth]{Immagini/Architettura_Software}
	\caption{Architettura Software}
	\label{fig:ArchitetturaSoftware}
\end{figure}
\newpage
Nella figura ~\ref{fig:ArchitetturaSoftware} è mostrata l'architettura software modellizzata attraverso una rete di Petri. Innanzitutto si può già osservare chè è stata definita seguendo il modello architetturale MVC:
\begin{itemize}
	\item A monte è prevista una parte riservata all'interfaccia grafica, attraverso la quale sarà possibile inviare e ricevere informazioni dal server applicativo. Si vede, infatti, che è prevista una comunicazione bidirezionale tra dispositivo Android e Server.
	\item Al centro sono rappresentate tutte le richieste a cui è in grado di rispondere. Queste, quindi, saranno funzioni implementate lato Server.
	\item Infine, è prevista una banca dati persistente, in questo caso un database relazione, al quale il Server Applicativo accede sia per operazioni di lettura che di scrittura, sempre con lo scopo di far fronte alle richieste provenienti a monte.
\end{itemize}
Si può quindi constatare che non si trattano di strati tra loro indipendenti, poichè il flusso dei dati li coinvolge tutti.
\section{Logical view}
\begin{figure}[h!]
	\centering
	\includegraphics[width=0.8\textwidth]{Immagini/Logical_View_part1}
	\caption{Logical view}
	\label{fig:LogicalView1}
\end{figure}
\begin{figure}[h!]
	\centering
	\includegraphics[width=0.8\textwidth]{Immagini/Logical_View_part2}
	\caption{Logical view}
	\label{fig:LogicalView2}
\end{figure}
\noindent
In figura ~\ref{fig:LogicalView1} e ~\ref{fig:LogicalView2} è mostrata la Logical View del sistema progettato. Si può osservare che segue il modello definito attraverso il pattern archietteturale Model View Controller/Presenter, dal momento che vengono individuati tre strati, ciascuno dei quali con le seguenti caratteristiche:
\begin{itemize}
	\item \textbf{Subsystem \textit{"GUI”:}} Rappresenta l’interfaccia grafica con la quale l’applicazione si presenterà. Ciascun componente fa riferimento ad una azione che può essere svolta attraverso lo smartphone, come l’accesso alla rete di Book Crossing(Login) o registrazione di un  libro. Questi componenti saranno quindi allocati direttamente sul dispositivo mobile. 
	\item \textbf{Subsystem \textit{"Request Manager”:}} Ha il compito di gestire le richieste provenienti da ciascun componente descritto nel subsystem \textit{GUI”}. Al suo interno sono indicati i componenti attraverso i quali si risponde alle chiamate provenienti dal client. Questo quindi descrive i componenti che saranno individuati sul server presente all’interno dell’architettura.
	\item \textbf{Subsystem \textit{"Data Manager":}} ”: Rappresenta la comunicazione con il Database. Sono quindi indicati i componenti con i quali il sistema si interfaccerà con la banca dati dell’architettura.
\end{itemize}
Il modello architetturale MVC è stato poi applicato anche successivamente per la progettazione delle componenti previste per ciascun elemento dell'architettura.

\newpage
\section{Parte algoritmica: \textit{reservation handler}}
La gestione delle prenotazioni dei libri è una delle parti innovative introdotte dalla start up: questo servizio mira a sfruttare la flessibilità della community di sharing, basata sull'idea dell'open-source, cercando comunque di offrire un servizio mirato ed attento alle necessità del lettore.

Ogni utente, purchè sia registrato all'interno del servizio di Book-sharing, può prenotare un determinato libro che si trova nello stato "Under reading", ovvero si trova in mano ad un altro utente, il quale lo sta leggendo.

Andiamo ad evidenziare gli attori coinvolti in questa operazione di prenotazione:
\begin{itemize}
	\item \textbf{\underline{Lettore} [L]:} esso rappresenta l'utente, registrato nella community, che possiede il libro oggetto della prenotazione. Indichiamo con:
	\begin{itemize}
		\item \textbf{{\LARGE $r_{L}$}:} raggio d'azione del lettore;
		\item \textbf{{\LARGE $ z_{0} $}:} zona di residenza (espressa come coordinate puntuali).
	\end{itemize}
	\item \textbf{\underline{Prenotanti} [$ P_{i}$ con $i=1,...,N $]:} rappresentano l'insieme degli N utenti, tutti interessati ad uno specifico libro in possesso dell'utente \textbf{L}.
	
	Oltre a questa informazione, ogni utente avrà fornito, al momento della registrazione, le seguenti informazioni:
	\begin{itemize}
		\item \textbf{{\LARGE $r^{P}_{i}$} con $i=1,...,N $:} raggio d'azione del lettore;
		\item \textbf{{\LARGE $z^{P}_{i}$} con $i=1,...,N $:} zona di residenza.
	\end{itemize}

	L'algoritmo può dunque essere scomposto in due macro-blocchi:
	\begin{itemize}
		\item \textbf{Step 0:} questa fase viene richiamata nel momento in cui il sistema inizia ad analizzare tutti gli utenti che hanno effettuato una prenotazione per un determinato libro che si trova nello stato di \textit{"Under reading"}.
		
		Tutti gli N prenotanti \textit{$ P_{i} $} vengono ordinati in base alla distanza dal lettore \textit{L}, indipendentemente da quello che è l'ordine temporale con cui è stata effettuata la prenotazione: la quantità di cui si terrà (come si può vedere nella figura ~\ref{fig:DistanceAlgorithm}) conto sarà quindi la distanza
		\begin{equation}
			|z^{p}_{i}-z_{0}|
		\end{equation}
		\begin{figure}[h!]
		\centering
		\includegraphics[width=0.8\textwidth]{Immagini/Algoritmo_Explanation}
		\caption{Distanza tra lettore e prenotante i-esimo}
		\label{fig:DistanceAlgorithm}
		\end{figure}
	
		Questo ordinamento corrisponde quindi sostanzialmente a creare una \textit{priority queue} in cui si va ad assegnare una maggiore priorità all'utente la cui zona di residenza è più vicina a quella del lettore in possesso del libro richiesto.
		
		\item \textbf{Step 1:} in questo macro-blocco andiamo effettivamente ad applicare l'algoritmo \textit{smart} per poter soddisfare, nella maniera migliore, le esigenze di ogni utente della community.
		
		
		L'idea di base è che, se utente lettore e utente prenotante hanno possibilità di incontrarsi, ovvero se i loro raggi d'azione si sovrappongono, essi potranno accordarsi direttamente sul luogo dello scambio, rendendo \textit{"safety"} il passaggio del libro: questo scambio avverrà ovviamente in una zona all'interno dell'intersezione dei raggi d'azione, come mostrato in figura ~\ref{fig:Zona_incontro}.
		
		\begin{figure}[h!]
			\centering
			\includegraphics[width=0.6\textwidth]{Immagini/Algorithm_PuntoIncontro.jpg}
			\caption{Zona d'incontro tra lettore e prenotante}
			\label{fig:Zona_incontro}
		\end{figure}
	
		Nel caso in cui invece, i due utenti interessati non abbiano la possibilità di trovare un luogo comune in cui potersi scambiare il libro fisicamente si avrà che, la rete di users appartenenti alla community farà da tramite, per portare il libro \textit{"coast-to-coast"}.
		
		Quindi, tramite un semplice pseudo-codice, possiamo descrivere il nostro algoritmo come
	
		\begin{algorithm}[H]
			\SetAlgoLined
			\KwData{Informazioni dell'utente lettore e di quello prenotante}
			\KwResult{Percorso ottimo dal lettore al prenotante}
			Step 0 (inizializzazione)\;
			\eIf{Distanza <= 0}{
				Trova un punto d'incontro nell'unione delle delle area;
				
				Notifica gli utenti di dove potersi scambiare direttamente il libro;
			}{
				Crea la rete di utenti che faranno da tramite tra lettore e prenotante;
				
				Ricercare il cammino ottimo (il libro si muoverà \textit{hand-to-hand});
			}
			\caption{Algoritmo di gestione della prenotazione}
		\end{algorithm}
		
		
		Nello specifico il calcolo della distanza avverrà tramite la funzionalità \textit{checkOverlap(Lettore, Prenotante)}, la quale andrà a verificare che:
		
		{\LARGE \begin{equation}
			|z^{p}_{i}-z_{0}|-r_{0}-r_{i}<=0
		\end{equation}}
		
		ovvero che i raggi d'azione si sovrappongano o meno.
		
		Nel caso in cui i due utenti non abbiano possibili punti d'incontro (distanza $>= 0 $), dobbiamo selezionare l'insieme di utenti tramite i quali il libro in questione potrà spostarsi: l'idea base è quindi quella di costruirsi un'area circolare di centro pari alla metà della congiungente del punto $ z_{0} $ (zona di residenza lettore) e $ z^{p}_{i} $ (zona di residenza prenotante).
		Verranno poi selezionati tutti gli utenti che si trovano all'interno di questa circoscrizione.
		
		In passi sequenziali, possiamo scrivere:
		
		\begin{algorithm}[H]
			\SetAlgoLined
			\KwData{Zona di residenza e raggio d'azione di utente lettore e prenotante}
			\KwResult{Elenco di utenti attraverso cui il libro dovrà spostarsi \textit{hand-to-hand}}
			
			
			Il raggio della circoscrizione di utenti coinvolti sarà pari alla distanza
			\[ \bar{Z} = \dfrac{1}{2} |z^{p}_{i}-z_{0}| \]
			\For{ogni utente $ z_{i}^U $ che si trova all'interno della community}{
			\If{$ (Distanza(z_{0},z_{i}^U)) <= \bar{Z} $ oppure $ (Distanza(z_{i}^P,z_{i}^U)) <= \bar{Z} $  }{
				
				Seleziono l'utente $ z_{i}^U $ e lo inserisco nella lista (\textit{HandToHandUsers}) dei possibili utenti che potrebbero partecipare attivamente al prestito;
			}
			}
			Creiamo il collegamento tra gli users della lista \textit{HandToHandUsers} il cui raggio d'azione si sovrappone\;
			\caption{Creazione del percorso tra lettore e prenotante}
		\end{algorithm}
		
		Quindi, alla fine dello step 1, avremo individuato tutti gli utenti i quali possono partecipare attivamente alla realizzazione di un prestito: il risultato ottenuto sarà quindi come quello in figura ~\ref{fig:UsersNet}.
						
		\begin{figure}[h!]
			\centering
			\includegraphics[width=0.8\textwidth]{Immagini/Algorithm_UsersNet}
			\caption{Rete di utenti che potrebbero essere attivi nel prestito}
			\label{fig:UsersNet}
		\end{figure}
		
	\end{itemize}
\end{itemize}

Un ulteriore miglioramento che è stato introdotto è rappresentato dal comportamento greedy dell'algoritmo \textit{decisionale}: come è facilmente intuibile, in questo contesto, l'obbiettivo base è quello di andare a minimizzare il numero di km percorsi dal libro, per giungere al lettore prenotante.

Il concetto è quindi quello di minimizzare il numero di utenti che prenderanno parte attivamente al prestito, realizzando un \textit{passamano} tra uno e l'altro.

Un approccio classico per minimizzare il percorso seguito dal libro durante il prestito è quello di andare a scegliere l'utente successivo a cui far arrivare il libro come l'utente la cui distanza è minima tra tutti quelli possibili.

Per rendere però più performante possiamo andare ad applicare un approccio \textit{greedy}, il quale consente di aumentare l'ottimalità dell'algoritmo stesso: nello specifico il concetto seguito nell'implementazione è stato quello di andare ad optare tra due differenti scelte decisionali, ovvero:
\begin{itemize}
	\item \textit{Probabilità $\epsilon$}: andiamo a scegliere l'utente più vicino, tra tutti quelli selezionabili;
	\item \textit{Probabilità $1 - \epsilon$}: tra gli utenti che si possono selezionare, si va a scegliere, con questa probabilità, l'utente con raggio d'azione maggiore
\end{itemize}

Grazie a questa duplice possibile scelta, siamo in grado di rendere l'algoritmo più ottimale, ovvero in grado di trovare, nella maggior parte dei casi, un percorso ottimo; in particolare, minore è il valore di $\epsilon$ che andiamo a scegliere, più \textit{forte} sarà l'algoritmo, ovvero troverà sempre un percorso ma senza garanzia che sia quello ottimale.
Maggiore invece è il valore di $\epsilon$ scelto, più bassa sarà la probabilità che troverà un percorso però, il percorso trovato, sarà uno dei più corti.
\end{document}
